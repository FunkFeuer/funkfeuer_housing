\documentclass[parskip=half]{scrreprt}



\usepackage[utf8]{inputenc}
\usepackage[T1]{fontenc}
\usepackage[ngerman]{babel}
\usepackage{lmodern}
\usepackage[juratotoc]{scrjura}
\usepackage[a4paper,margin=1.5cm,top=1.6cm,bottom=2cm,includeheadfoot,headsep=1.3cm]{geometry}
\usepackage{pdfpages}
\usepackage{lastpage}
\usepackage{fancyhdr} % headers
\usepackage{graphicx} % logo include
\usepackage{color} % text colors
\usepackage{setspace}

\usepackage{paratype} % font
\renewcommand*\familydefault{\sfdefault} %% Only if the base font of the document is to be sans serif

\def\Address{
    FunkFeuer Wien - Verein zur Förderung freier Netze\\
    c/o Volkskundemuseum Wien\\
    Laudongasse 15-19\\
    1080 Wien\\
}

% FunkFeuer Colors
\definecolor{ff-dark}{RGB}{39,49,66}
\definecolor{ff-grey}{RGB}{85,98,112}
\definecolor{ff-light}{RGB}{105,145,170}
\definecolor{ff-red}{RGB}{242,58,20}
\definecolor{ff-green}{RGB}{161,200,32}

\usepackage{wrapfig}

\pagestyle{fancy}

\fancyhf{}
\fancyhead[R]{\includegraphics[width=3.9cm]{funkfeuer_wien.pdf}}
\fancyhead[L]{\textcolor{ff-grey}{\textbf{Verein zur Förderung freier Netze}\\\small vorstand@funkfeuer.at}}
\fancyfoot[L]{
    \begin{wrapfigure}{L}{1.6cm}
        \vspace{-8.5pt}
        \includegraphics[width=1.75cm]{logo.pdf}
    \end{wrapfigure}
    \vspace{-5pt}
    \fontsize{7}{8} \selectfont \textcolor{ff-grey}{
        \\
        \Address
}}
\fancyfoot[R]{
    \vspace{-2pt}
    \fontsize{7}{8} \selectfont \textcolor{ff-grey}{
        ZVR: 814804682\\
        UID: ATU67830859\\
        BIC/SWIFT: SPLSAT21\\
        IBAN: AT552023000000143982\\
}}
\fancyfoot[C]{
    \textcolor{ff-grey}{
    \thepage/\pageref{LastPage}
}}


% lines
\renewcommand{\headrulewidth}{2pt}
\renewcommand{\headrule}{\hbox to\headwidth{%
  \color{ff-green}\leaders\hrule height \headrulewidth\hfill}}
\renewcommand{\footrulewidth}{0.5pt}
\renewcommand{\footrule}{\hbox to\headwidth{%
  \color{ff-light}\leaders\hrule height \footrulewidth\hfill}}
% section colors
\usepackage{sectsty} % autocolor sections
\chapterfont{\color{ff-grey}}  % sets colour of chapters
\sectionfont{\color{ff-grey}}  % sets colour of sections
% Text color
\color{ff-dark}

\def\CustomerNo{000}
\def\ServerNo{000}
\def\CustomerCompany{}
\def\CustomerName{Vorname Nachmane}
\def\CustomerStreet{Straße 1}
\def\CustomerCity{1000 Wien}
\def\CustomerCountry{Österreich}

\def\MandateRef{\CustomerNo}


\begin{document}
\small
\chapter*{Nutzungsvertrag Serverhousing Stellplatz}
\thispagestyle{fancy}
\textbf{Gehäuseformat Mini-ITX\\ \\
Kundennummer: \CustomerNo \\
Servernummer: \ServerNo
}

\vspace{0.5cm}
Abgeschlossen zwischen

\begin{tabular}{p{15cm}p{0.5cm}l}
\Address
\end{tabular}

als Projekt housing.funkfeuer.at einerseits und

\begin{tabular}{p{15cm}p{0.5cm}l}
\CustomerName\\
\CustomerStreet\\
\CustomerCity
\end{tabular}

als Serverbetreiber andererseits wie folgt:
%\vspace{0.5cm}

\begin{contract}
   \begin{enumerate}
\item Der Serverbetreiber ist alleinig für den Inhalt auf seinem Rechner verantwortlich.
\item Der Serverbetreiber stellt sicher, dass der Betrieb des Servers und darauf laufende Dienste geltendes österreichisches Recht nicht verletzen.
\item Der Serverbetreiber hat einen elektrischen Leistungsbezug und eine Datenmenge nach dem Fair Use Prinzip. Für die faire Verteilung der Leistungen sind Geräte mit maximal 45 Watt Leistungsaufnahme und 100 GB Datenverbrauch wünschenswert. Zusätzlich Datenmengen können nur in Absprache mit der Leitung des Projektes housing.funkfeuer.at vereinbart werden.
\item Für den Leistungsbezug sind 19 Euro Unkostenbeitrag pro Monat zu bezahlen. Der Betrag ist per Einziehungsauftrag im Voraus zu entrichten oder für 6 Monate im Voraus per Banküberweisung zu bezahlen. Bei Zahlungsverzug von über zwei Wochen ist das Projekt housing.funkfeuer.at berechtigt, die Leistungen einzustellen. Die offenen Forderungen bleiben bestehen.
\item Der Serverbetreiber hat nach Terminvereinbarung uneingeschränkten Zugang zu seinem Rechner.
\item Der Serverbetreiber bestätigt dass er die AGBs des Projektes housing.funkfeuer.at zur Kenntnis genommen hat, und sie uneingeschränkt akzeptiert.
\item Der Serverbetreiber bekommt eine öffentliche weltweit zugängliche IP Adresse zugeteilt. Es gibt keine Einschränkung von Serverdiensten, jedoch ist das in Punkt 3 erwähnte Fair Use Prinzip einzuhalten. Letzteres gilt besonders für Peer-to-Peer Dienste, wie Bit Torrent.
\item Der Serverbetreiber sorgt dafür, dass sein Rechner automatisch hochfahren kann.
\item Das Projekt housing.funkfeuer.at haftet keinesfalls, weder bei Datenverlust noch für die Hardware, auch nicht falls dies durch den laufenden Betrieb des Projektes housing.funkfeuer.at verursacht wurde.
\item Dieser Vertrag kann zu jedem Monatsende, unter Einhaltung einer vierwöchigen Kündigungsfrist, gekündigt werden.
\end{enumerate}
\end{contract}

\vspace{0.5cm}
\begin{tabular}{p{7cm}p{0.5cm}l}
\dotfill \\
Serverbetreiber
\end{tabular}
\hfill
\begin{tabular}{p{7cm}p{0.5cm}l}
\dotfill \\
Obfrau
\end{tabular}

\vspace{0.5cm}
\begin{tabular}{p{7cm}p{0.5cm}l}
\dotfill \\
Ort, Datum
\end{tabular}
\hfill
\begin{tabular}{p{7cm}p{0.5cm}l}
\dotfill \\
Schriftführer
\end{tabular}


\includepdf[pages=-]{sepa_mandat.pdf}

\end{document}
