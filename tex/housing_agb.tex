\documentclass[parskip=half]{article}

\usepackage[utf8]{inputenc}
\usepackage[T1]{fontenc}
\usepackage[ngerman]{babel}
\usepackage{lmodern}
\usepackage[juratotoc]{scrjura}
%\usepackage[margin=2cm]{geometry}
%\usepackage{geometry}
% \usepackage[textheight=8in,footskip=60pt,bottom=1in]{geometry}
\usepackage[a4paper,margin=1.5cm,top=1.6cm,bottom=4cm]{geometry}
\usepackage{lastpage}
\usepackage{fancyhdr}% headers
\usepackage{graphicx} % logo include
\usepackage{color} % text colors
\usepackage{setspace}

\usepackage{paratype} % font
\renewcommand*\familydefault{\sfdefault} %% Only if the base font of the document is to be sans serif
\usepackage[T1]{fontenc}
% FunkFeuer Colors
\definecolor{ff-dark}{RGB}{39,49,66}
\definecolor{ff-grey}{RGB}{85,98,112}
\definecolor{ff-light}{RGB}{105,145,170}
\definecolor{ff-red}{RGB}{242,58,20}
\definecolor{ff-green}{RGB}{161,200,32}



\usepackage{wrapfig}

% header
\pagestyle{fancy}
\fancyhf{}
\fancyhead[R]{\includegraphics[width=3.9cm]{funkfeuer_wien.pdf}}
%\fancyhead[L]{\textcolor{ff-grey}{\textbf{Verein zur Förderung freier Netze} - ZVR: 814804682} \\
%	\textcolor{ff-grey}{Postfach 44, 1016 Wien, Österreich}}
%\fancyhead[L]{\textcolor{ff-grey}{\textbf{Verein zur Förderung freier Netze}\\ZVR: 814804682}
\fancyhead[L]{\textcolor{ff-grey}{\textbf{Verein zur Förderung freier Netze}\\\small admin@funkfeuer.at}
}

% footer
%\fancyfoot[L]{\vskip -10pt~\includegraphics[width=1.5cm]{logo.pdf}}
% \fancyfoot[C]{\textcolor{ff-dark}{FunkFeuer Wien - ZVR: 814804682 \\Postfach 5 · 1096 Wien \\ IBAN: AT552023000000143982 · BIC/SWIFT: SPLSAT21}}
\fancyfoot[L]{
\begin{wrapfigure}{L}{1.6cm}
\vspace{-15pt}
\includegraphics[width=1.75cm]{logo.pdf}
\end{wrapfigure}
\vspace{-5pt}
\fontsize{7}{8} \selectfont
\textcolor{ff-grey}{\hspace*{-1mm}Postanschrift:\\FunkFeuer Wien\\ Postfach 44\\1016 Wien}
}
\fancyfoot[C]{\fontsize{7}{8} \selectfont \textcolor{ff-grey}{\\ZVR: 814804682\\UID: ATU67830859\\BIC/SWIFT: SPLSAT21\\ IBAN: AT552023000000143982}
}
\fancyfoot[R]{\textcolor{ff-grey}{\thepage/\pageref{LastPage}}}

% lines
\renewcommand{\headrulewidth}{2pt}
\renewcommand{\headrule}{\hbox to\headwidth{%
 \color{ff-green}\leaders\hrule height \headrulewidth\hfill}}
\renewcommand{\footrulewidth}{0.5pt}
\renewcommand{\footrule}{\hbox to\headwidth{%
 \color{ff-light}\leaders\hrule height \footrulewidth\hfill}}
% section colors
\usepackage{sectsty} % autocolor sections
\chapterfont{\color{ff-grey}} % sets colour of chapters
\sectionfont{\color{ff-grey}} % sets colour of sections
% Text color
\color{ff-dark}

%
%	EDIT HERE
%

\def\CustomerNo{000}
\def\ServerNo{000}
\def\CustomerCompany{}
\def\CustomerName{Vorname Nachmane}
\def\CustomerStreet{Straße 1}
\def\CustomerCity{1000 Wien}
\def\CustomerCountry{Österreich}

\def\MandateRef{\CustomerNo}


\renewcommand{\thesection}{\Roman{section}}
\renewcommand{\thesubsection}{\thesection.\Roman{subsection}}

\begin{document}
\thispagestyle{fancy}
\section*{\\Allgemeine Geschäftsbedingungen für den Serverbereich \\ des
Projektes housing.funkfeuer.at}

\subsection{Umfang und Geltungsbereich}
\begin{enumerate}
\item Die allgemeinen Geschäfts- und Lieferbedingungen vom Projekt
housing.funkfeuer.at gelten für alle Dienstleistungen und Lieferungen, die das
Projekt housing.funkfeuer.at dem Auftraggeber gegenüber erbringt. Sie gelten
auch für alle zukünftigen Verträge, selbst wenn nicht ausdrücklich darauf
Bezug genommen wird.
\item In Ergänzung der Allgemeinen Geschäfts- und Lieferbedingungen vom
2.3.2006 gelten die Allgemeinen Bedingungen für Dienstleistungen in der
Informationsverarbeitung durch Rechenzentren, herausgegeben vom Fachverband
Unternehmensberatung und Datenverarbeitung, Wirtschaftskammer Oesterreich,
in der aktuellen Fassung.
\end{enumerate}

\subsection{ Rechtsvorschriften}
\begin{enumerate}
\item Der Vertragspartner verpflichtet sich, die österreichischen Gesetze auch im
internationalen Datenverkehr über das Projekt housing.funkfeuer.at einzuhalten und
bemerkte Gesetzesverstoesse dem Projekt housing.funkfeuer.at zu melden.
\item Der Vertragspartner verpflichtet sich weiters, bei Verstößen gegen österreichische
oder internationale Gesetze (wie insbesondere dem Telekommunikationsgesetz, dem
Mediengesetz, dem Verbotsgesetz, dem Pornographiegesetz, dem Urheberrechtsgesetz,
dem Strafgesetzbuch), das Projekt housing.funkfeuer.at von jedem Nachteil freizuhalten,
der durch von Vertragspartnern übermittelnden, verbreitenden oder ausgestellten Daten
und Nachrichten entsteht, und das Projekt housing.funkfeuer.at schad- und klaglos zu
halten.
\item Der Vertragspartner ist verschuldensunabhängig verantwortlich für sämtliche
Aktivitaeten, die von seinem Anschluss ausgehen und das Projekt housing.funkfeuer.at für
sämtliche entstehenden Schäden schad und klaglos halten. Von der vollkommenen Schad- und
Klagloshaltung sind insbesondere auch zu zahlende Strafen welcher Art auch immer und die
Kosten einer zweckentsprechenden Rechtsverteidigung erfasst.
\end{enumerate}

\subsection{Vertragsbeginn und Vertragsdauer}
\begin{enumerate}
\item Eine Kündigung beider Vertragspatner ist jeweils unter Einhaltung einer 14 tägigen
Kündigungsfrist möglich.
\item Die in Katalogen, Aussendungen und dergleichen enthaltenen Angaben sind nur
massgeblich, wenn in der Auftragsbestätigung ausdrücklich auf sie Bezug genommen
wird.
\end{enumerate}

\subsection{Entgeltentrichtung}
\begin{enumerate}
\item Pro Monat ist ein im Nutzungsvertrag festgelegter Unkostenbeitrag zu entrichten. Mit dem
Betrag bezahlt das Projekt housing.funkfeuer.at die Betriebskosten, bestehend aus
Stromkosten, Wartungskosten, Infrastrukturkosten und Leitungskosten.
\end{enumerate}

\subsection{ Haftungsausschluss}
\begin{enumerate}
\item Das Projekt housing.funkfeuer.at haftet für Schäden ausserhalb des
Anwendungsbereiches des Produkthaftungsgesetzes nur, sofern ihm Vorsatz oder grobe
Fahrlässigkeit nachgewiesen werden, im Rahmen der gesetzlichen Vorschriften. Die
Haftung für leichte Fahrlaessigkeit, der Ersatz von Folgeschäden und Vermoegensschäden, nicht erzielten Ersparnissen, entgangenem Gewinn, verlorengegangene Daten, Zinsverlusten und
von Schäden aus Ansprüchen Dritter gegen den Vertragspartnern sind ausgeschlossen. Insbesondere sind jegliche Ansprüche bei Ausfall von Datenleitungen oder der Projekthousing.funkfeuer.at Server ausgeschlossen.
\item Das Projekt housing.funkfeuer.at haftet nicht für Inhalt, Vollständigkeit, Richtigkeit usw.
übermittelter oder abgefragter Daten und für Daten, die über das Projekt
housing.funkfeuer.at erreichbar sind.
\item Das Projekt housing.funkfeuer.at übernimmt keine Gewähr, dass die angebotenen
Dienste immer zugänglich sind und dass auf den Rechnern vom Projekt
housing.funkfeuer.at gespeicherte Daten immer erhalten bleiben.
\item Das Projekt housing.funkfeuer.at behält sich das Recht vor, einzelne öffentlich
zugängliche Angebote zu sperren, wenn dies Rechtsvorschriften erfordern. Oder wenn es
zur Aufrechterhaltung des störungsfreien Betriebs erforderlich ist.
\item Das Projekt housing.funkfeuer.at übernimmt keine Haftung für Schäden, die durch
eine erforderliche, aber nicht erteilte fernmeldebehoerdliche Bewilligung oder andere
behördliche Genehmigungen oder durch erforderliche, aber nicht erteilte privatrechtliche
Genehmigungen oder Zustimmung Dritter entstehen.
\end{enumerate}

\subsection{ Datenschutz}
\begin{enumerate}
\item Das Projekt housing.funkfeuer.at ist berechtigt, Verbindungsdaten, insbesondere Source-
und Destination-IP und sämtliche anderen Logfiles neben der Auswertung für
Verrechnungszwecke, zum Schutz der eigenen Rechner und der von Dritten, zu speichern
und auszuwerten. Weiters dürfen diese Daten zur Behebung technischer Maengel
verwendet werden.
\item Das Projekt housing.funkfeuer.at ist berechtigt, Stammdaten der Vertragspartnern und
Teilnehmer, wie Titel, Vornamen, Nachnamen, Geburtsdatum, Firma, Adresse, Branche,
Bankverbindung, E-Mail Adresse, Anfragedatum, schriftlich geregelte übereinkünfte
speichern. Diese Stammdaten werden automationsunterstützt verarbeitet und ohne
schriftliche Zustimmung des Teilnehmers nicht weitergegeben. Entsprechend der in dem
Paragraph 96 Telekommunikationsgesetz (TKG) normierten Verpflichtung erstellt das
Projekt housing.funkfeuer.at ein auf Web basierendes Teilnehmerverzeichnis, in dem der
Vertragspartner Vor- und Familienname, Titel, Berufsbezeichnung, Adresse, E-Mail-Adresse
und weitere Daten eintragen kann. Auf Wunsch des Vertragspartnern kann diese
Eintragung unterbleiben. Über das technisch notwendige Mindestmass werden
Inhaltsdaten jedoch nicht gespeichert und keinesfalls ausgewertet. Das Projekt
housing.funkfeuer.at ist berechtigt, Access-Statistiken zu führen.
\item Das Projekt housing.funkfeuer.at ergreift alle technisch möglichen Massnahmen, um die
bei ihm gespeicherten Vertragspartnerndaten zu schützen. Das Projekt
housing.funkfeuer.at haftet jedoch nicht, wenn sich Dritte auf rechtswidrige Art und Weise
diese Daten in ihre Verfügungsgewalt bringen und sie weiterverwenden. Die
Geltendmachung von Schäden der Vertragspartei oder Dritter gegenüber dem Projekt
housing.funkfeuer.at aus einem derartigen Zusammenhang wird einvernehmlich
ausgeschlossen.
\end{enumerate}

\subsection{Software}
\begin{enumerate}
\item für Software, die als eFreeware, ePublic Domain, eDemo oder als eShareware
klassifiziert ist, übernimmt das Projekt housing.funkfeuer.at keine wie immer geartete
Gewähr. Die vom jeweiligen Programmautor für diese Software angegebenen
Nutzungsbestimmungen oder allfaellige Lizenzregelungen sind zu beachten.
\item Das Projekt housing.funkfeuer.at übernimmt keine Gewähr dafür, dass die Software
jederzeit und fehlerfrei funktioniert und mit anderen Programmen oder
Hardwarezusammensetzungen zusammenarbeitet.
\end{enumerate}

\subsection{IP-Vergabe}
\begin{enumerate}
\item Der vom Projekt housing.funkfeuer.at zugewiesene IP Bereich darf nur vom Auftraggeber
(Endbenutzer) genutzt werden. Die Weitergabe an Dritte ist weder zur Gänze noch zu
Teilen möglich. Das Projekt housing.funkfeuer.at behält sich das Recht, IP Adressen an
Dritte weiterzugeben, vor.
\item Die IP Adressen dürfen nur während eines aufrechten Vertrages mit dem Projekt
housing.funkfeuer.at genutzt werden. Endet das Vertragsverhältnis, aus welchem Grunde
auch immer, sind die IP Adressen an das Projekt housing.funkfeuer.at zurückzugeben.
\item Das Projekt housing.funkfeuer.at ist bei der Vergabe von IP Adressen an internationale
Richtlinien gebunden. Somit gilt insbesondere:
\begin{enumerate}
\item Pro Jahr und Benutzer kann nur eine bestimmte IP Adressen Anzahl vergeben werden.
\item Sollten sich von Benutzer angegebene Informationen als ungültig herausstellen, ist die
Zuweisung nicht länger gültig.
\item IP-Adressen dürfen nicht auf Vorrat registriert werden.
\end{enumerate}
\item Es kann auf Grund von technischer Notwendigkeit vorkommen, dass ein zugewiesener
Adressbereich durch einen anderen ersetzt werden muss. Das Projekt housing.funkfeuer.at
kann daher die Verwendung von bestimmten IP Adressen nicht garantieren.
\end{enumerate}

\subsection{ Rücktritt}
\begin{enumerate}
\item[] Das Projekt housing.funkfeuer.at ist berechtigt, vom Vertrag zurückzutreten, wenn
\item\begin{enumerate}
\item der Nutzer einen im Verhaeltnis zu dem mit ihm vereinbarten Datenvolumen
überproportionalen Datentransfer aufweist oder der Nutzer Dienste übermässig in
Anspruch nimmt;
\item der Nutzer wiederholt gegen die eNetiquettee und die allgemein akzeptierten Standards
der Netzbenutzung verstößt, wie auch durch ungebetenes Werben und Spamming
(aggressives Direct-Mailing), die Benutzung des Dienstes zur übertragung von
Drohungen, Obszoenitäten, Belästigungen oder zur Schädigung anderer Teilnehmer.
\end{enumerate}
\item Unbeschadet der Schadenersatzansprüche vom Projekt housing.funkfeuer.at sind im
Falle des Rücktritts bereits erbrachte Leistungen oder Teilleistungen vertragsgemäß
abzurechnen und zu bezahlen.
\end{enumerate}

\subsection{Nettiquette}
\begin{enumerate}
\item Der Vertragspartner verpflichtet sich, die international ueblichen Verhaltensregeln einzuhalten: \\
Ärgere andere Netzteilnehmer nicht übermaessig und ärgere dich über andere
Netzteilnehmer nicht übermässig. Sollten aus dem Internet diesbezueglich Beschwerden über
den Vertragspartner an das Projekt housing.funkfeuer.at herangetragen werden, so ist das Projekt
housing.funkfeuer.at im Wiederholungsfall berechtigt, das Vertragsverhältnis aufzulösen.
\item Bei Zuwiderhandeln kann das Projekt housing.funkfeuer.at den Zugang des
Vertragspartners ohne Angabe von Gründen und ohne vorherige Information des
Vertragspartners sperren.
\end{enumerate}

\subsection{ Änderungen der Allgemeinen Geschäftsbedingungen und der Entgelte}
\begin{enumerate}
\item[] Änderungen der Allgemeinen Geschaeftsbestimmungen und der Entgelte werden dem
Vertragspartnern schriftlich (per E-Mail) mitgeteilt. Die Änderungen gelten als akzeptiert,
wenn der Vertragspartner diesen nicht innerhalb von 30 Tagen nach Aussendung der
Mitteilung schriftlich (per E-Mail) widerspricht. Der Widerspruch gilt als Kündigung.
\end{enumerate}

\subsection{ Sonstige Bestimmungen}
\begin{enumerate}
\item Alle dieses Vertragsverhältnis betreffenden Änderungen, Ergänzungen, Mitteilungen
und Erklärungen sind nur gültig, wenn sie schriftlich oder per E-Mail erfolgen bei
sonstiger Unwirksamkeit. Mündliche Nebenabreden bestehen nicht.
\item Digitale Unterschriften vom Projekt housing.funkfeuer.at werden als rechtsgültig
anerkannt.
\item Das Projekt housing.funkfeuer.at ist auf eigenes Risiko ermächtigt, andere
Unternehmen mit der Erbringung von Leistungen aus diesem Vertragsverhältnis zu
beauftragen.
\item Der Vertragspartner wird Änderungen seines Namens oder der Bezeichnung, die er
beim Projekt housing.funkfeuer.at angegeben hat, sowie jede Änderung seiner Anschrift
(Sitzverlegung) oder seiner Rechtsform sofort, spätestens jedoch innerhalb eines Monats
ab der Änderung anzeigen. Gibt der Vertragspartner solche Änderungen nicht bekannt
und gehen ihm deshalb an die von ihm zuletzt bekanntgegebene Anschrift gesandte,
rechtlich bedeutsame Erklaerungen vom Projekt housing.funkfeuer.at, insbesondere
Rechnungen, Mahnungen oder Kündigungen nicht zu, so gelten diese Erklärungen vom
Projekt housing.funkfeuer.at trotzdem als zugegangen.
\item Der Vertrag bleibt auch bei rechtlicher Unwirksamkeit einzelner Regelungen und
Bedingungen in seinen übrigen Teilen wirksam. Das gilt nicht, wenn in diesem Falle das
Festhalten an dem Vertrag eine unzumutbare Härte für eine Vertragspartei darstellen
würde.
\item Der Vertragspartner verpflichtet sich, seinen Zugang zum Projekt housing.funkfeuer.at
und die damit verbundenen Dienstleistungen nicht an Dritte weiterzugeben und sein
Passwort geheim zu halten. Für Schäden, die durch mangelhafte Geheimhaltung der
Passwörter durch den Vertragspartner oder durch Weitergabe an Dritte entstehen, haftet
dieser. Vergessene Zugangs-Passwörter werden vom Projekt housing.funkfeuer.at auf
Verlangen unentgeltlich durch neue ersetzt. Jeder Verdacht einer unerlaubten Benutzung
seines Zuganges durch Dritte muss dem Projekt housing.funkfeuer.at sofort gemeldet
werden.
\item für die Kommunikation zwischen Vertragspartnern und dem Projekt housing.funkfeuer.at ist,
soweit möglich, E-Mail zu verwenden.
\end{enumerate}

\subsection{ Gerichtsstand}
\begin{enumerate}
\item Als Gerichtsstand gilt Wien als vereinbart, ausser bei Klagen gegen Verbraucher im Sinne
des Konsumentenschutzgesetzes, die ihren Wohnsitz oder gewöhnlichen Aufenthalt im
Inland haben oder im Inland beschäftigt sind. Es gilt ausschließlich österreichisches Recht.
\end{enumerate}

\end{document}
