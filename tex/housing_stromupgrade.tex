\documentclass[parskip=half]{scrreprt}



\usepackage[utf8]{inputenc}
\usepackage[T1]{fontenc}
\usepackage[ngerman]{babel}
\usepackage{lmodern}
\usepackage[juratotoc]{scrjura}
\usepackage[a4paper,margin=1.5cm,top=1.6cm,bottom=2cm,includeheadfoot,headsep=1.3cm]{geometry}
\usepackage{pdfpages}
\usepackage{lastpage}
\usepackage{fancyhdr} % headers
\usepackage{graphicx} % logo include
\usepackage{color} % text colors
\usepackage{setspace}

\usepackage{paratype} % font
\renewcommand*\familydefault{\sfdefault} %% Only if the base font of the document is to be sans serif

\def\Address{
    FunkFeuer Wien - Verein zur Förderung freier Netze\\
    c/o Volkskundemuseum Wien\\
    Laudongasse 15-19\\
    1080 Wien\\
}

% FunkFeuer Colors
\definecolor{ff-dark}{RGB}{39,49,66}
\definecolor{ff-grey}{RGB}{85,98,112}
\definecolor{ff-light}{RGB}{105,145,170}
\definecolor{ff-red}{RGB}{242,58,20}
\definecolor{ff-green}{RGB}{161,200,32}

\usepackage{wrapfig}

\pagestyle{fancy}

\fancyhf{}
\fancyhead[R]{\includegraphics[width=3.9cm]{funkfeuer_wien.pdf}}
\fancyhead[L]{\textcolor{ff-grey}{\textbf{Verein zur Förderung freier Netze}\\\small vorstand@funkfeuer.at}}
\fancyfoot[L]{
    \begin{wrapfigure}{L}{1.6cm}
        \vspace{-8.5pt}
        \includegraphics[width=1.75cm]{logo.pdf}
    \end{wrapfigure}
    \vspace{-5pt}
    \fontsize{7}{8} \selectfont \textcolor{ff-grey}{
        \\
        \Address
}}
\fancyfoot[R]{
    \vspace{-2pt}
    \fontsize{7}{8} \selectfont \textcolor{ff-grey}{
        ZVR: 814804682\\
        UID: ATU67830859\\
        BIC/SWIFT: SPLSAT21\\
        IBAN: AT552023000000143982\\
}}
\fancyfoot[C]{
    \textcolor{ff-grey}{
    \thepage/\pageref{LastPage}
}}


% lines
\renewcommand{\headrulewidth}{2pt}
\renewcommand{\headrule}{\hbox to\headwidth{%
  \color{ff-green}\leaders\hrule height \headrulewidth\hfill}}
\renewcommand{\footrulewidth}{0.5pt}
\renewcommand{\footrule}{\hbox to\headwidth{%
  \color{ff-light}\leaders\hrule height \footrulewidth\hfill}}
% section colors
\usepackage{sectsty} % autocolor sections
\chapterfont{\color{ff-grey}}  % sets colour of chapters
\sectionfont{\color{ff-grey}}  % sets colour of sections
% Text color
\color{ff-dark}

\def\CustomerNo{000}
\def\ServerNo{000}
\def\CustomerCompany{}
\def\CustomerName{Vorname Nachmane}
\def\CustomerStreet{Straße 1}
\def\CustomerCity{1000 Wien}
\def\CustomerCountry{Österreich}

\def\MandateRef{\CustomerNo}


\begin{document}
\chapter*{Vertrag über Bezug zusätzlicher elektrischer Leistung}
\thispagestyle{fancy}
\textbf{Kundennummer: \CustomerNo \\
Servernummer: \ServerNo
}

\vspace{0.5cm}
Abgeschlossen zwischen

\begin{tabular}{p{15cm}p{0.5cm}l}
FunkFeuer Wien - Verein zur Förderung freier Netze (ZVR 814804682)\\
 c/o Volkskundemuseum Wien\\
 Laudongasse 15-19\\
 1080 Wien
\end{tabular}

als Projekt housing.funkfeuer.at einerseits und

\begin{tabular}{p{15cm}p{0.5cm}l}
\CustomerName\\
\CustomerStreet\\
\CustomerCity
\end{tabular}

als Serverbetreiber andererseits wie folgt:
\vspace{0.5cm}

\begin{contract}
   \begin{enumerate}
\item Der Serverbetreiber bucht für seine beim Projekt housing.funkfeuer.at eingestellten Server zusätzliche, über die normale Fair-Use Vereinbarung laut Server Housing Nutzungsvertrag hinausgehende, elektrische Leistung von
\vspace{0.2cm}\\
150 (3*50) Watt pro Monat.
\vspace{0.2cm}\\
Für diesen zusätzlichen Leistungsbezug sind 18 Euro Unkostenbeitrag pro Monat zu bezahlen.
\item Der Serverbetreiber bestätigt dass er die AGBs des Projektes housing.funkfeuer.at zur Kenntnis genommen hat, und Sie uneingeschränkt akzeptiert.
\item Dieser Vertrag kann zu jedem Monatsende, unter Einhaltung einer vierwöchigen Kündigungsfrist, gekündigt werden.
\end{enumerate}
\end{contract}

\vspace{1,5 cm}
\begin{tabular}{p{7cm}p{.5cm}l}
\dotfill \\
Serverbetreiber
\end{tabular}
\hfill
\begin{tabular}{p{7cm}p{.5cm}l}
\dotfill \\
Obmann
\end{tabular}

\vspace{0.5cm}
\begin{tabular}{p{7cm}p{.5cm}l}
\dotfill \\
Ort, Datum
\end{tabular}
\hfill
\begin{tabular}{p{7cm}p{.5cm}l}
\dotfill \\
Schriftführer
\end{tabular}

\end{document}
